% Options for packages loaded elsewhere
% Options for packages loaded elsewhere
\PassOptionsToPackage{unicode}{hyperref}
\PassOptionsToPackage{hyphens}{url}
\PassOptionsToPackage{dvipsnames,svgnames,x11names}{xcolor}
%
\documentclass[
  letterpaper,
  DIV=11,
  numbers=noendperiod]{scrartcl}
\usepackage{xcolor}
\usepackage{amsmath,amssymb}
\setcounter{secnumdepth}{-\maxdimen} % remove section numbering
\usepackage{iftex}
\ifPDFTeX
  \usepackage[T1]{fontenc}
  \usepackage[utf8]{inputenc}
  \usepackage{textcomp} % provide euro and other symbols
\else % if luatex or xetex
  \usepackage{unicode-math} % this also loads fontspec
  \defaultfontfeatures{Scale=MatchLowercase}
  \defaultfontfeatures[\rmfamily]{Ligatures=TeX,Scale=1}
\fi
\usepackage{lmodern}
\ifPDFTeX\else
  % xetex/luatex font selection
\fi
% Use upquote if available, for straight quotes in verbatim environments
\IfFileExists{upquote.sty}{\usepackage{upquote}}{}
\IfFileExists{microtype.sty}{% use microtype if available
  \usepackage[]{microtype}
  \UseMicrotypeSet[protrusion]{basicmath} % disable protrusion for tt fonts
}{}
\makeatletter
\@ifundefined{KOMAClassName}{% if non-KOMA class
  \IfFileExists{parskip.sty}{%
    \usepackage{parskip}
  }{% else
    \setlength{\parindent}{0pt}
    \setlength{\parskip}{6pt plus 2pt minus 1pt}}
}{% if KOMA class
  \KOMAoptions{parskip=half}}
\makeatother
% Make \paragraph and \subparagraph free-standing
\makeatletter
\ifx\paragraph\undefined\else
  \let\oldparagraph\paragraph
  \renewcommand{\paragraph}{
    \@ifstar
      \xxxParagraphStar
      \xxxParagraphNoStar
  }
  \newcommand{\xxxParagraphStar}[1]{\oldparagraph*{#1}\mbox{}}
  \newcommand{\xxxParagraphNoStar}[1]{\oldparagraph{#1}\mbox{}}
\fi
\ifx\subparagraph\undefined\else
  \let\oldsubparagraph\subparagraph
  \renewcommand{\subparagraph}{
    \@ifstar
      \xxxSubParagraphStar
      \xxxSubParagraphNoStar
  }
  \newcommand{\xxxSubParagraphStar}[1]{\oldsubparagraph*{#1}\mbox{}}
  \newcommand{\xxxSubParagraphNoStar}[1]{\oldsubparagraph{#1}\mbox{}}
\fi
\makeatother

\usepackage{color}
\usepackage{fancyvrb}
\newcommand{\VerbBar}{|}
\newcommand{\VERB}{\Verb[commandchars=\\\{\}]}
\DefineVerbatimEnvironment{Highlighting}{Verbatim}{commandchars=\\\{\}}
% Add ',fontsize=\small' for more characters per line
\usepackage{framed}
\definecolor{shadecolor}{RGB}{241,243,245}
\newenvironment{Shaded}{\begin{snugshade}}{\end{snugshade}}
\newcommand{\AlertTok}[1]{\textcolor[rgb]{0.68,0.00,0.00}{#1}}
\newcommand{\AnnotationTok}[1]{\textcolor[rgb]{0.37,0.37,0.37}{#1}}
\newcommand{\AttributeTok}[1]{\textcolor[rgb]{0.40,0.45,0.13}{#1}}
\newcommand{\BaseNTok}[1]{\textcolor[rgb]{0.68,0.00,0.00}{#1}}
\newcommand{\BuiltInTok}[1]{\textcolor[rgb]{0.00,0.23,0.31}{#1}}
\newcommand{\CharTok}[1]{\textcolor[rgb]{0.13,0.47,0.30}{#1}}
\newcommand{\CommentTok}[1]{\textcolor[rgb]{0.37,0.37,0.37}{#1}}
\newcommand{\CommentVarTok}[1]{\textcolor[rgb]{0.37,0.37,0.37}{\textit{#1}}}
\newcommand{\ConstantTok}[1]{\textcolor[rgb]{0.56,0.35,0.01}{#1}}
\newcommand{\ControlFlowTok}[1]{\textcolor[rgb]{0.00,0.23,0.31}{\textbf{#1}}}
\newcommand{\DataTypeTok}[1]{\textcolor[rgb]{0.68,0.00,0.00}{#1}}
\newcommand{\DecValTok}[1]{\textcolor[rgb]{0.68,0.00,0.00}{#1}}
\newcommand{\DocumentationTok}[1]{\textcolor[rgb]{0.37,0.37,0.37}{\textit{#1}}}
\newcommand{\ErrorTok}[1]{\textcolor[rgb]{0.68,0.00,0.00}{#1}}
\newcommand{\ExtensionTok}[1]{\textcolor[rgb]{0.00,0.23,0.31}{#1}}
\newcommand{\FloatTok}[1]{\textcolor[rgb]{0.68,0.00,0.00}{#1}}
\newcommand{\FunctionTok}[1]{\textcolor[rgb]{0.28,0.35,0.67}{#1}}
\newcommand{\ImportTok}[1]{\textcolor[rgb]{0.00,0.46,0.62}{#1}}
\newcommand{\InformationTok}[1]{\textcolor[rgb]{0.37,0.37,0.37}{#1}}
\newcommand{\KeywordTok}[1]{\textcolor[rgb]{0.00,0.23,0.31}{\textbf{#1}}}
\newcommand{\NormalTok}[1]{\textcolor[rgb]{0.00,0.23,0.31}{#1}}
\newcommand{\OperatorTok}[1]{\textcolor[rgb]{0.37,0.37,0.37}{#1}}
\newcommand{\OtherTok}[1]{\textcolor[rgb]{0.00,0.23,0.31}{#1}}
\newcommand{\PreprocessorTok}[1]{\textcolor[rgb]{0.68,0.00,0.00}{#1}}
\newcommand{\RegionMarkerTok}[1]{\textcolor[rgb]{0.00,0.23,0.31}{#1}}
\newcommand{\SpecialCharTok}[1]{\textcolor[rgb]{0.37,0.37,0.37}{#1}}
\newcommand{\SpecialStringTok}[1]{\textcolor[rgb]{0.13,0.47,0.30}{#1}}
\newcommand{\StringTok}[1]{\textcolor[rgb]{0.13,0.47,0.30}{#1}}
\newcommand{\VariableTok}[1]{\textcolor[rgb]{0.07,0.07,0.07}{#1}}
\newcommand{\VerbatimStringTok}[1]{\textcolor[rgb]{0.13,0.47,0.30}{#1}}
\newcommand{\WarningTok}[1]{\textcolor[rgb]{0.37,0.37,0.37}{\textit{#1}}}

\usepackage{longtable,booktabs,array}
\usepackage{calc} % for calculating minipage widths
% Correct order of tables after \paragraph or \subparagraph
\usepackage{etoolbox}
\makeatletter
\patchcmd\longtable{\par}{\if@noskipsec\mbox{}\fi\par}{}{}
\makeatother
% Allow footnotes in longtable head/foot
\IfFileExists{footnotehyper.sty}{\usepackage{footnotehyper}}{\usepackage{footnote}}
\makesavenoteenv{longtable}
\usepackage{graphicx}
\makeatletter
\newsavebox\pandoc@box
\newcommand*\pandocbounded[1]{% scales image to fit in text height/width
  \sbox\pandoc@box{#1}%
  \Gscale@div\@tempa{\textheight}{\dimexpr\ht\pandoc@box+\dp\pandoc@box\relax}%
  \Gscale@div\@tempb{\linewidth}{\wd\pandoc@box}%
  \ifdim\@tempb\p@<\@tempa\p@\let\@tempa\@tempb\fi% select the smaller of both
  \ifdim\@tempa\p@<\p@\scalebox{\@tempa}{\usebox\pandoc@box}%
  \else\usebox{\pandoc@box}%
  \fi%
}
% Set default figure placement to htbp
\def\fps@figure{htbp}
\makeatother





\setlength{\emergencystretch}{3em} % prevent overfull lines

\providecommand{\tightlist}{%
  \setlength{\itemsep}{0pt}\setlength{\parskip}{0pt}}



 


\KOMAoption{captions}{tableheading}
\makeatletter
\@ifpackageloaded{caption}{}{\usepackage{caption}}
\AtBeginDocument{%
\ifdefined\contentsname
  \renewcommand*\contentsname{Table of contents}
\else
  \newcommand\contentsname{Table of contents}
\fi
\ifdefined\listfigurename
  \renewcommand*\listfigurename{List of Figures}
\else
  \newcommand\listfigurename{List of Figures}
\fi
\ifdefined\listtablename
  \renewcommand*\listtablename{List of Tables}
\else
  \newcommand\listtablename{List of Tables}
\fi
\ifdefined\figurename
  \renewcommand*\figurename{Figure}
\else
  \newcommand\figurename{Figure}
\fi
\ifdefined\tablename
  \renewcommand*\tablename{Table}
\else
  \newcommand\tablename{Table}
\fi
}
\@ifpackageloaded{float}{}{\usepackage{float}}
\floatstyle{ruled}
\@ifundefined{c@chapter}{\newfloat{codelisting}{h}{lop}}{\newfloat{codelisting}{h}{lop}[chapter]}
\floatname{codelisting}{Listing}
\newcommand*\listoflistings{\listof{codelisting}{List of Listings}}
\makeatother
\makeatletter
\makeatother
\makeatletter
\@ifpackageloaded{caption}{}{\usepackage{caption}}
\@ifpackageloaded{subcaption}{}{\usepackage{subcaption}}
\makeatother
\usepackage{bookmark}
\IfFileExists{xurl.sty}{\usepackage{xurl}}{} % add URL line breaks if available
\urlstyle{same}
\hypersetup{
  pdftitle={Prática 07 - Teste Baseado em Propriedades},
  pdfauthor={Breno Farias da Silva},
  colorlinks=true,
  linkcolor={blue},
  filecolor={Maroon},
  citecolor={Blue},
  urlcolor={Blue},
  pdfcreator={LaTeX via pandoc}}


\title{Prática 07 - Teste Baseado em Propriedades}
\author{Breno Farias da Silva}
\date{2025-06-09}
\begin{document}
\maketitle


\subsection{Introdução}\label{introduuxe7uxe3o}

Este relatório descreve a aplicação da técnica de \textbf{teste baseado
em propriedades (Property-Based Testing)} para verificar a implementação
de uma função que valida identificadores na linguagem \emph{Silly
Pascal}. A atividade faz parte da disciplina \textbf{PPGCC12 - Teste de
Software}.

A função \texttt{validateIdentifier(String\ s)} possui o objetivo de
determinar se um identificador é válido segundo as seguintes regras:

\begin{itemize}
\tightlist
\item
  Deve começar com uma letra (maiúscula ou minúscula).
\item
  Pode conter apenas letras e dígitos após o primeiro caractere.
\item
  O comprimento deve ser de \textbf{1 a 6 caracteres}, inclusive.
\end{itemize}

\subsection{Análise da Solução
Implementada}\label{anuxe1lise-da-soluuxe7uxe3o-implementada}

A análise do código fornecido revela alguns problemas. O trecho abaixo
mostra a implementação:

\begin{Shaded}
\begin{Highlighting}[]
\KeywordTok{public} \DataTypeTok{boolean} \FunctionTok{validateIdentifier}\OperatorTok{(}\BuiltInTok{String}\NormalTok{ s}\OperatorTok{)} \OperatorTok{\{}
  \DataTypeTok{char}\NormalTok{ achar}\OperatorTok{;}
  \DataTypeTok{boolean}\NormalTok{ valid\_id }\OperatorTok{=} \KeywordTok{false}\OperatorTok{;}
\NormalTok{  achar }\OperatorTok{=}\NormalTok{ s}\OperatorTok{.}\FunctionTok{charAt}\OperatorTok{(}\DecValTok{0}\OperatorTok{);}
\NormalTok{  valid\_id }\OperatorTok{=} \OperatorTok{((}\NormalTok{achar }\OperatorTok{\textgreater{}=} \CharTok{\textquotesingle{}A\textquotesingle{}}\OperatorTok{)} \OperatorTok{\&\&} \OperatorTok{(}\NormalTok{achar }\OperatorTok{\textless{}=} \CharTok{\textquotesingle{}Z\textquotesingle{}}\OperatorTok{))} \OperatorTok{||} \OperatorTok{((}\NormalTok{achar }\OperatorTok{\textgreater{}=} \CharTok{\textquotesingle{}a\textquotesingle{}}\OperatorTok{)} \OperatorTok{\&\&} \OperatorTok{(}\NormalTok{achar }\OperatorTok{\textless{}=} \CharTok{\textquotesingle{}z\textquotesingle{}}\OperatorTok{));}

  \ControlFlowTok{if} \OperatorTok{(}\NormalTok{s}\OperatorTok{.}\FunctionTok{length}\OperatorTok{()} \OperatorTok{\textgreater{}} \DecValTok{1}\OperatorTok{)} \OperatorTok{\{}
\NormalTok{      achar }\OperatorTok{=}\NormalTok{ s}\OperatorTok{.}\FunctionTok{charAt}\OperatorTok{(}\DecValTok{1}\OperatorTok{);}
      \DataTypeTok{int}\NormalTok{ i }\OperatorTok{=} \DecValTok{1}\OperatorTok{;}
      \ControlFlowTok{while} \OperatorTok{(}\NormalTok{i }\OperatorTok{\textless{}}\NormalTok{ s}\OperatorTok{.}\FunctionTok{length}\OperatorTok{()} \OperatorTok{{-}} \DecValTok{1}\OperatorTok{)} \OperatorTok{\{}
\NormalTok{          achar }\OperatorTok{=}\NormalTok{ s}\OperatorTok{.}\FunctionTok{charAt}\OperatorTok{(}\NormalTok{i}\OperatorTok{);}
          \ControlFlowTok{if} \OperatorTok{(((}\NormalTok{achar }\OperatorTok{\textgreater{}=} \CharTok{\textquotesingle{}A\textquotesingle{}}\OperatorTok{)} \OperatorTok{\&\&} \OperatorTok{(}\NormalTok{achar }\OperatorTok{\textless{}=} \CharTok{\textquotesingle{}Z\textquotesingle{}}\OperatorTok{))} \OperatorTok{||} \OperatorTok{((}\NormalTok{achar }\OperatorTok{\textgreater{}=} \CharTok{\textquotesingle{}a\textquotesingle{}}\OperatorTok{)} \OperatorTok{\&\&} \OperatorTok{(}\NormalTok{achar }\OperatorTok{\textless{}=} \CharTok{\textquotesingle{}z\textquotesingle{}}\OperatorTok{))} \OperatorTok{||} \OperatorTok{((}\NormalTok{achar }\OperatorTok{\textgreater{}=} \CharTok{\textquotesingle{}0\textquotesingle{}}\OperatorTok{)} \OperatorTok{\&\&} \OperatorTok{(}\NormalTok{achar }\OperatorTok{\textless{}=} \CharTok{\textquotesingle{}9\textquotesingle{}}\OperatorTok{)))} \OperatorTok{\{}
\NormalTok{              valid\_id }\OperatorTok{=} \KeywordTok{false}\OperatorTok{;}
          \OperatorTok{\}}
\NormalTok{          i}\OperatorTok{++;}
      \OperatorTok{\}}
  \OperatorTok{\}}
  \ControlFlowTok{if} \OperatorTok{(}\NormalTok{valid\_id }\OperatorTok{\&\&} \OperatorTok{(}\NormalTok{s}\OperatorTok{.}\FunctionTok{length}\OperatorTok{()} \OperatorTok{\textgreater{}=} \DecValTok{1}\OperatorTok{)} \OperatorTok{\&\&} \OperatorTok{(}\NormalTok{s}\OperatorTok{.}\FunctionTok{length}\OperatorTok{()} \OperatorTok{\textless{}} \DecValTok{6}\OperatorTok{))}
      \ControlFlowTok{return} \KeywordTok{true}\OperatorTok{;}
  \ControlFlowTok{else}
      \ControlFlowTok{return} \KeywordTok{false}\OperatorTok{;}
\OperatorTok{\}}
\end{Highlighting}
\end{Shaded}

\subsubsection{\texorpdfstring{\textbf{Observações sobre o
código}}{Observações sobre o código}}\label{observauxe7uxf5es-sobre-o-cuxf3digo}

\begin{itemize}
\tightlist
\item
  Há um erro lógico no laço \texttt{while}: a condição invalida o
  identificador sempre que um caractere válido (letra ou dígito) é
  encontrado. Isso é oposto ao esperado.
\item
  O limite de tamanho está incorreto, pois o código verifica
  \texttt{s.length()\ \textless{}\ 6}, quando deveria ser
  \texttt{\textless{}=\ 6}.
\item
  Não há tratamento para strings vazias, o que pode gerar exceção
  (\texttt{StringIndexOutOfBoundsException}).
\end{itemize}

\subsection{Definição das
Propriedades}\label{definiuxe7uxe3o-das-propriedades}

Baseado na especificação, definimos as seguintes propriedades para a
função:

\subsubsection{\texorpdfstring{\textbf{Propriedades de entrada
válida}}{Propriedades de entrada válida}}\label{propriedades-de-entrada-vuxe1lida}

\begin{itemize}
\tightlist
\item
  Para qualquer string que:

  \begin{itemize}
  \tightlist
  \item
    Tenha de 1 a 6 caracteres;
  \item
    O primeiro caractere seja uma letra (\texttt{{[}a-zA-Z{]}});
  \item
    Os demais caracteres (se existirem) sejam letras ou dígitos
    (\texttt{{[}a-zA-Z0-9{]}});
  \end{itemize}
\end{itemize}

A função \textbf{deve retornar \texttt{true}}.

\subsubsection{\texorpdfstring{\textbf{Propriedades de entrada
inválida}}{Propriedades de entrada inválida}}\label{propriedades-de-entrada-invuxe1lida}

\begin{itemize}
\tightlist
\item
  A função deve retornar \textbf{\texttt{false}} se:

  \begin{itemize}
  \tightlist
  \item
    O primeiro caractere não for uma letra;
  \item
    O comprimento for menor que 1 ou maior que 6;
  \item
    Contiver qualquer caractere que não seja letra ou dígito.
  \end{itemize}
\end{itemize}

\subsection{Implementação dos Testes Baseados em
Propriedades}\label{implementauxe7uxe3o-dos-testes-baseados-em-propriedades}

A implementação dos testes foi realizada utilizando o framework
\textbf{jqwik} para Java. Foram criados três conjuntos de propriedades:

\subsubsection{\texorpdfstring{\textbf{Testa identificadores
válidos}}{Testa identificadores válidos}}\label{testa-identificadores-vuxe1lidos}

\begin{Shaded}
\begin{Highlighting}[]
\AttributeTok{@Property}
\DataTypeTok{void} \FunctionTok{validIdentifiers}\OperatorTok{(}\AttributeTok{@ForAll} \AttributeTok{@AlphaChars} \AttributeTok{@StringLength}\OperatorTok{(}\NormalTok{min }\OperatorTok{=} \DecValTok{0}\OperatorTok{,}\NormalTok{ max }\OperatorTok{=} \DecValTok{5}\OperatorTok{)} \BuiltInTok{String}\NormalTok{ suffix}\OperatorTok{)} \OperatorTok{\{}
    \BuiltInTok{String}\NormalTok{ id }\OperatorTok{=} \StringTok{"A"} \OperatorTok{+}\NormalTok{ suffix}\OperatorTok{;}
    \FunctionTok{assumeTrue}\OperatorTok{(}\NormalTok{id}\OperatorTok{.}\FunctionTok{length}\OperatorTok{()} \OperatorTok{\textless{}=} \DecValTok{6}\OperatorTok{);}
    \FunctionTok{assertThat}\OperatorTok{(}\NormalTok{identifier}\OperatorTok{.}\FunctionTok{validateIdentifier}\OperatorTok{(}\NormalTok{id}\OperatorTok{)).}\FunctionTok{isTrue}\OperatorTok{();}
\OperatorTok{\}}
\end{Highlighting}
\end{Shaded}

\subsubsection{\texorpdfstring{\textbf{Testa identificadores inválidos
por
tamanho}}{Testa identificadores inválidos por tamanho}}\label{testa-identificadores-invuxe1lidos-por-tamanho}

\begin{Shaded}
\begin{Highlighting}[]
\AttributeTok{@Property}
\DataTypeTok{void} \FunctionTok{invalidLengthIdentifiers}\OperatorTok{(}\AttributeTok{@ForAll} \AttributeTok{@StringLength}\OperatorTok{(}\NormalTok{min }\OperatorTok{=} \DecValTok{7}\OperatorTok{,}\NormalTok{ max }\OperatorTok{=} \DecValTok{20}\OperatorTok{)} \BuiltInTok{String}\NormalTok{ id}\OperatorTok{)} \OperatorTok{\{}
    \FunctionTok{assertThat}\OperatorTok{(}\NormalTok{identifier}\OperatorTok{.}\FunctionTok{validateIdentifier}\OperatorTok{(}\NormalTok{id}\OperatorTok{)).}\FunctionTok{isFalse}\OperatorTok{();}
\OperatorTok{\}}
\end{Highlighting}
\end{Shaded}

\subsubsection{\texorpdfstring{\textbf{Testa identificadores inválidos
por
caracteres}}{Testa identificadores inválidos por caracteres}}\label{testa-identificadores-invuxe1lidos-por-caracteres}

\begin{Shaded}
\begin{Highlighting}[]
\AttributeTok{@Property}
\DataTypeTok{void} \FunctionTok{invalidCharacterIdentifiers}\OperatorTok{(}\AttributeTok{@ForAll} \AttributeTok{@StringLength}\OperatorTok{(}\NormalTok{min }\OperatorTok{=} \DecValTok{0}\OperatorTok{,}\NormalTok{ max }\OperatorTok{=} \DecValTok{5}\OperatorTok{)} \BuiltInTok{String}\NormalTok{ suffix}\OperatorTok{)} \OperatorTok{\{}
    \BuiltInTok{String}\NormalTok{ id }\OperatorTok{=} \StringTok{"1"} \OperatorTok{+}\NormalTok{ suffix}\OperatorTok{;}
    \FunctionTok{assumeTrue}\OperatorTok{(}\NormalTok{id}\OperatorTok{.}\FunctionTok{length}\OperatorTok{()} \OperatorTok{\textless{}=} \DecValTok{6}\OperatorTok{);}
    \FunctionTok{assertThat}\OperatorTok{(}\NormalTok{identifier}\OperatorTok{.}\FunctionTok{validateIdentifier}\OperatorTok{(}\NormalTok{id}\OperatorTok{)).}\FunctionTok{isFalse}\OperatorTok{();}
\OperatorTok{\}}
\end{Highlighting}
\end{Shaded}

\subsection{Resultados dos Testes}\label{resultados-dos-testes}

\begin{longtable}[]{@{}
  >{\raggedright\arraybackslash}p{(\linewidth - 4\tabcolsep) * \real{0.5342}}
  >{\raggedright\arraybackslash}p{(\linewidth - 4\tabcolsep) * \real{0.2466}}
  >{\raggedright\arraybackslash}p{(\linewidth - 4\tabcolsep) * \real{0.2192}}@{}}
\toprule\noalign{}
\begin{minipage}[b]{\linewidth}\raggedright
Propriedade
\end{minipage} & \begin{minipage}[b]{\linewidth}\raggedright
Status
\end{minipage} & \begin{minipage}[b]{\linewidth}\raggedright
Casos Gerados
\end{minipage} \\
\midrule\noalign{}
\endhead
\bottomrule\noalign{}
\endlastfoot
Identificadores válidos & Falha & 1 (primeiro caso) \\
Identificadores inválidos por tamanho & Passou & 1000 \\
Identificadores inválidos por caract. & Passou & 1000 \\
\end{longtable}

\subsubsection{\texorpdfstring{\textbf{Detalhes da
falha}}{Detalhes da falha}}\label{detalhes-da-falha}

\begin{itemize}
\tightlist
\item
  A propriedade \textbf{``Identificadores válidos''} falhou logo na
  primeira execução.
\item
  Caso que falhou: \texttt{"AA"} e \texttt{"YxU"}.
\item
  Erro: A função retornou \texttt{false} para identificadores válidos.
\end{itemize}

\subsection{Cobertura de Código}\label{cobertura-de-cuxf3digo}

\begin{longtable}[]{@{}ll@{}}
\toprule\noalign{}
Métrica & Cobertura \\
\midrule\noalign{}
\endhead
\bottomrule\noalign{}
\endlastfoot
Linha & 100\% \\
Branch (Decisão) & 100\% \\
Condição & 100\% \\
\end{longtable}

\subsection{Conclusão}\label{conclusuxe3o}

O teste baseado em propriedades se mostrou eficiente na identificação de
problemas na implementação. A falha detectada reforça que a função não
atende corretamente à especificação quando se trata de validar
identificadores válidos.

A técnica não só verifica casos simples, mas também explora
automaticamente cenários de borda e aleatórios, proporcionando uma
cobertura ampla e eficaz.

\subsubsection{\texorpdfstring{🔧
\textbf{Recomendação}}{🔧 Recomendação}}\label{recomendauxe7uxe3o}

A implementação fornecida apresenta erros lógicos. Recomenda-se a sua
correção conforme a especificação. Segue uma versão correta da função:

\begin{Shaded}
\begin{Highlighting}[]
\KeywordTok{public} \DataTypeTok{boolean} \FunctionTok{validateIdentifier}\OperatorTok{(}\BuiltInTok{String}\NormalTok{ s}\OperatorTok{)} \OperatorTok{\{}
    \ControlFlowTok{if} \OperatorTok{(}\NormalTok{s }\OperatorTok{==} \KeywordTok{null} \OperatorTok{||}\NormalTok{ s}\OperatorTok{.}\FunctionTok{length}\OperatorTok{()} \OperatorTok{\textless{}} \DecValTok{1} \OperatorTok{||}\NormalTok{ s}\OperatorTok{.}\FunctionTok{length}\OperatorTok{()} \OperatorTok{\textgreater{}} \DecValTok{6}\OperatorTok{)}
        \ControlFlowTok{return} \KeywordTok{false}\OperatorTok{;}

    \ControlFlowTok{if} \OperatorTok{(!}\BuiltInTok{Character}\OperatorTok{.}\FunctionTok{isLetter}\OperatorTok{(}\NormalTok{s}\OperatorTok{.}\FunctionTok{charAt}\OperatorTok{(}\DecValTok{0}\OperatorTok{)))}
        \ControlFlowTok{return} \KeywordTok{false}\OperatorTok{;}

    \ControlFlowTok{for} \OperatorTok{(}\DataTypeTok{int}\NormalTok{ i }\OperatorTok{=} \DecValTok{1}\OperatorTok{;}\NormalTok{ i }\OperatorTok{\textless{}}\NormalTok{ s}\OperatorTok{.}\FunctionTok{length}\OperatorTok{();}\NormalTok{ i}\OperatorTok{++)} \OperatorTok{\{}
        \DataTypeTok{char}\NormalTok{ c }\OperatorTok{=}\NormalTok{ s}\OperatorTok{.}\FunctionTok{charAt}\OperatorTok{(}\NormalTok{i}\OperatorTok{);}
        \ControlFlowTok{if} \OperatorTok{(!}\BuiltInTok{Character}\OperatorTok{.}\FunctionTok{isLetterOrDigit}\OperatorTok{(}\NormalTok{c}\OperatorTok{))}
            \ControlFlowTok{return} \KeywordTok{false}\OperatorTok{;}
    \OperatorTok{\}}

    \ControlFlowTok{return} \KeywordTok{true}\OperatorTok{;}
\OperatorTok{\}}
\end{Highlighting}
\end{Shaded}

\subsection{Referências}\label{referuxeancias}

\begin{itemize}
\tightlist
\item
  Livro: \emph{Effective Software Testing} --- Maurício Aniche, 2022.
\item
  Documentação jqwik: https://jqwik.net/docs/current/user-guide.html
\end{itemize}




\end{document}
