% Options for packages loaded elsewhere
% Options for packages loaded elsewhere
\PassOptionsToPackage{unicode}{hyperref}
\PassOptionsToPackage{hyphens}{url}
\PassOptionsToPackage{dvipsnames,svgnames,x11names}{xcolor}
%
\documentclass[
  letterpaper,
  DIV=11,
  numbers=noendperiod]{scrartcl}
\usepackage{xcolor}
\usepackage{amsmath,amssymb}
\setcounter{secnumdepth}{-\maxdimen} % remove section numbering
\usepackage{iftex}
\ifPDFTeX
  \usepackage[T1]{fontenc}
  \usepackage[utf8]{inputenc}
  \usepackage{textcomp} % provide euro and other symbols
\else % if luatex or xetex
  \usepackage{unicode-math} % this also loads fontspec
  \defaultfontfeatures{Scale=MatchLowercase}
  \defaultfontfeatures[\rmfamily]{Ligatures=TeX,Scale=1}
\fi
\usepackage{lmodern}
\ifPDFTeX\else
  % xetex/luatex font selection
\fi
% Use upquote if available, for straight quotes in verbatim environments
\IfFileExists{upquote.sty}{\usepackage{upquote}}{}
\IfFileExists{microtype.sty}{% use microtype if available
  \usepackage[]{microtype}
  \UseMicrotypeSet[protrusion]{basicmath} % disable protrusion for tt fonts
}{}
\makeatletter
\@ifundefined{KOMAClassName}{% if non-KOMA class
  \IfFileExists{parskip.sty}{%
    \usepackage{parskip}
  }{% else
    \setlength{\parindent}{0pt}
    \setlength{\parskip}{6pt plus 2pt minus 1pt}}
}{% if KOMA class
  \KOMAoptions{parskip=half}}
\makeatother
% Make \paragraph and \subparagraph free-standing
\makeatletter
\ifx\paragraph\undefined\else
  \let\oldparagraph\paragraph
  \renewcommand{\paragraph}{
    \@ifstar
      \xxxParagraphStar
      \xxxParagraphNoStar
  }
  \newcommand{\xxxParagraphStar}[1]{\oldparagraph*{#1}\mbox{}}
  \newcommand{\xxxParagraphNoStar}[1]{\oldparagraph{#1}\mbox{}}
\fi
\ifx\subparagraph\undefined\else
  \let\oldsubparagraph\subparagraph
  \renewcommand{\subparagraph}{
    \@ifstar
      \xxxSubParagraphStar
      \xxxSubParagraphNoStar
  }
  \newcommand{\xxxSubParagraphStar}[1]{\oldsubparagraph*{#1}\mbox{}}
  \newcommand{\xxxSubParagraphNoStar}[1]{\oldsubparagraph{#1}\mbox{}}
\fi
\makeatother

\usepackage{color}
\usepackage{fancyvrb}
\newcommand{\VerbBar}{|}
\newcommand{\VERB}{\Verb[commandchars=\\\{\}]}
\DefineVerbatimEnvironment{Highlighting}{Verbatim}{commandchars=\\\{\}}
% Add ',fontsize=\small' for more characters per line
\newenvironment{Shaded}{}{}
\newcommand{\AlertTok}[1]{\textcolor[rgb]{1.00,0.00,0.00}{\textbf{#1}}}
\newcommand{\AnnotationTok}[1]{\textcolor[rgb]{0.38,0.63,0.69}{\textbf{\textit{#1}}}}
\newcommand{\AttributeTok}[1]{\textcolor[rgb]{0.49,0.56,0.16}{#1}}
\newcommand{\BaseNTok}[1]{\textcolor[rgb]{0.25,0.63,0.44}{#1}}
\newcommand{\BuiltInTok}[1]{\textcolor[rgb]{0.00,0.50,0.00}{#1}}
\newcommand{\CharTok}[1]{\textcolor[rgb]{0.25,0.44,0.63}{#1}}
\newcommand{\CommentTok}[1]{\textcolor[rgb]{0.38,0.63,0.69}{\textit{#1}}}
\newcommand{\CommentVarTok}[1]{\textcolor[rgb]{0.38,0.63,0.69}{\textbf{\textit{#1}}}}
\newcommand{\ConstantTok}[1]{\textcolor[rgb]{0.53,0.00,0.00}{#1}}
\newcommand{\ControlFlowTok}[1]{\textcolor[rgb]{0.00,0.44,0.13}{\textbf{#1}}}
\newcommand{\DataTypeTok}[1]{\textcolor[rgb]{0.56,0.13,0.00}{#1}}
\newcommand{\DecValTok}[1]{\textcolor[rgb]{0.25,0.63,0.44}{#1}}
\newcommand{\DocumentationTok}[1]{\textcolor[rgb]{0.73,0.13,0.13}{\textit{#1}}}
\newcommand{\ErrorTok}[1]{\textcolor[rgb]{1.00,0.00,0.00}{\textbf{#1}}}
\newcommand{\ExtensionTok}[1]{#1}
\newcommand{\FloatTok}[1]{\textcolor[rgb]{0.25,0.63,0.44}{#1}}
\newcommand{\FunctionTok}[1]{\textcolor[rgb]{0.02,0.16,0.49}{#1}}
\newcommand{\ImportTok}[1]{\textcolor[rgb]{0.00,0.50,0.00}{\textbf{#1}}}
\newcommand{\InformationTok}[1]{\textcolor[rgb]{0.38,0.63,0.69}{\textbf{\textit{#1}}}}
\newcommand{\KeywordTok}[1]{\textcolor[rgb]{0.00,0.44,0.13}{\textbf{#1}}}
\newcommand{\NormalTok}[1]{#1}
\newcommand{\OperatorTok}[1]{\textcolor[rgb]{0.40,0.40,0.40}{#1}}
\newcommand{\OtherTok}[1]{\textcolor[rgb]{0.00,0.44,0.13}{#1}}
\newcommand{\PreprocessorTok}[1]{\textcolor[rgb]{0.74,0.48,0.00}{#1}}
\newcommand{\RegionMarkerTok}[1]{#1}
\newcommand{\SpecialCharTok}[1]{\textcolor[rgb]{0.25,0.44,0.63}{#1}}
\newcommand{\SpecialStringTok}[1]{\textcolor[rgb]{0.73,0.40,0.53}{#1}}
\newcommand{\StringTok}[1]{\textcolor[rgb]{0.25,0.44,0.63}{#1}}
\newcommand{\VariableTok}[1]{\textcolor[rgb]{0.10,0.09,0.49}{#1}}
\newcommand{\VerbatimStringTok}[1]{\textcolor[rgb]{0.25,0.44,0.63}{#1}}
\newcommand{\WarningTok}[1]{\textcolor[rgb]{0.38,0.63,0.69}{\textbf{\textit{#1}}}}

\usepackage{longtable,booktabs,array}
\usepackage{calc} % for calculating minipage widths
% Correct order of tables after \paragraph or \subparagraph
\usepackage{etoolbox}
\makeatletter
\patchcmd\longtable{\par}{\if@noskipsec\mbox{}\fi\par}{}{}
\makeatother
% Allow footnotes in longtable head/foot
\IfFileExists{footnotehyper.sty}{\usepackage{footnotehyper}}{\usepackage{footnote}}
\makesavenoteenv{longtable}
\usepackage{graphicx}
\makeatletter
\newsavebox\pandoc@box
\newcommand*\pandocbounded[1]{% scales image to fit in text height/width
  \sbox\pandoc@box{#1}%
  \Gscale@div\@tempa{\textheight}{\dimexpr\ht\pandoc@box+\dp\pandoc@box\relax}%
  \Gscale@div\@tempb{\linewidth}{\wd\pandoc@box}%
  \ifdim\@tempb\p@<\@tempa\p@\let\@tempa\@tempb\fi% select the smaller of both
  \ifdim\@tempa\p@<\p@\scalebox{\@tempa}{\usebox\pandoc@box}%
  \else\usebox{\pandoc@box}%
  \fi%
}
% Set default figure placement to htbp
\def\fps@figure{htbp}
\makeatother





\setlength{\emergencystretch}{3em} % prevent overfull lines

\providecommand{\tightlist}{%
  \setlength{\itemsep}{0pt}\setlength{\parskip}{0pt}}



 


\KOMAoption{captions}{tableheading}
\makeatletter
\@ifpackageloaded{caption}{}{\usepackage{caption}}
\AtBeginDocument{%
\ifdefined\contentsname
  \renewcommand*\contentsname{Table of contents}
\else
  \newcommand\contentsname{Table of contents}
\fi
\ifdefined\listfigurename
  \renewcommand*\listfigurename{List of Figures}
\else
  \newcommand\listfigurename{List of Figures}
\fi
\ifdefined\listtablename
  \renewcommand*\listtablename{List of Tables}
\else
  \newcommand\listtablename{List of Tables}
\fi
\ifdefined\figurename
  \renewcommand*\figurename{Figure}
\else
  \newcommand\figurename{Figure}
\fi
\ifdefined\tablename
  \renewcommand*\tablename{Table}
\else
  \newcommand\tablename{Table}
\fi
}
\@ifpackageloaded{float}{}{\usepackage{float}}
\floatstyle{ruled}
\@ifundefined{c@chapter}{\newfloat{codelisting}{h}{lop}}{\newfloat{codelisting}{h}{lop}[chapter]}
\floatname{codelisting}{Listing}
\newcommand*\listoflistings{\listof{codelisting}{List of Listings}}
\makeatother
\makeatletter
\makeatother
\makeatletter
\@ifpackageloaded{caption}{}{\usepackage{caption}}
\@ifpackageloaded{subcaption}{}{\usepackage{subcaption}}
\makeatother
\usepackage{bookmark}
\IfFileExists{xurl.sty}{\usepackage{xurl}}{} % add URL line breaks if available
\urlstyle{same}
\hypersetup{
  pdftitle={Relatório Trabalho Prático de Teste Baseado em Especificação},
  pdfauthor={Breno Farias da Silva},
  colorlinks=true,
  linkcolor={blue},
  filecolor={Maroon},
  citecolor={Blue},
  urlcolor={Blue},
  pdfcreator={LaTeX via pandoc}}


\title{Relatório Trabalho Prático de Teste Baseado em Especificação}
\usepackage{etoolbox}
\makeatletter
\providecommand{\subtitle}[1]{% add subtitle to \maketitle
  \apptocmd{\@title}{\par {\large #1 \par}}{}{}
}
\makeatother
\subtitle{Validação do método preenche(String str, int size, String
stringPreenche)}
\author{Breno Farias da Silva}
\date{}
\begin{document}
\maketitle

\renewcommand*\contentsname{Sumário}
{
\hypersetup{linkcolor=}
\setcounter{tocdepth}{3}
\tableofcontents
}

\subsection{Introdução}\label{introduuxe7uxe3o}

Este relatório descreve o processo de teste da função \texttt{preenche},
responsável por preencher uma string à esquerda até atingir um tamanho
especificado, utilizando uma string de preenchimento fornecida. O método
recebe três parâmetros: a string original (\texttt{str}), o tamanho
desejado (\texttt{size}) e a string de preenchimento
(\texttt{stringPreenche}). O relatório é dividido entre os testes
baseados em especificação e os testes estruturais utilizando o critério
MC/DC, além da análise de cobertura com a ferramenta Jacoco.

\subsection{Teste baseado em
especificação}\label{teste-baseado-em-especificauxe7uxe3o}

\subsubsection{Análise da
especificação}\label{anuxe1lise-da-especificauxe7uxe3o}

A especificação define os seguintes comportamentos:

\begin{itemize}
\tightlist
\item
  Se \texttt{str} for \texttt{null}, o método retorna \texttt{null}.
\item
  Se \texttt{stringPreenche} for \texttt{null} ou vazia, deve ser
  tratada como um espaço em branco.
\item
  Se o tamanho de \texttt{str} for maior ou igual a \texttt{size}, o
  método retorna \texttt{str}.
\item
  Caso contrário, a string é preenchida à esquerda com
  \texttt{stringPreenche} até atingir o tamanho \texttt{size}.
\end{itemize}

\subsubsection{Identificação de partições de
equivalência}\label{identificauxe7uxe3o-de-partiuxe7uxf5es-de-equivaluxeancia}

\begin{itemize}
\tightlist
\item
  \texttt{str}:

  \begin{itemize}
  \tightlist
  \item
    \texttt{null} → retorna \texttt{null}
  \item
    não nula:

    \begin{itemize}
    \tightlist
    \item
      vazia: \texttt{""}
    \item
      1 caractere: \texttt{"x"}
    \item
      múltiplos caracteres: \texttt{"abc"}
    \end{itemize}
  \end{itemize}
\item
  \texttt{stringPreenche}:

  \begin{itemize}
  \tightlist
  \item
    \texttt{null} ou \texttt{""} → tratado como \texttt{"\ "}
  \item
    não vazia:

    \begin{itemize}
    \tightlist
    \item
      1 caractere: \texttt{"*"}
    \item
      mais de 1 caractere: \texttt{"-="}
    \end{itemize}
  \end{itemize}
\item
  \texttt{size}:

  \begin{itemize}
  \tightlist
  \item
    menor que \texttt{str.length()} → retorna \texttt{str}
  \item
    igual a \texttt{str.length()} → retorna \texttt{str}
  \item
    maior que \texttt{str.length()}:

    \begin{itemize}
    \tightlist
    \item
      diferença de 1 caractere
    \item
      diferença de múltiplos caracteres (para avaliar padrão de
      repetição)
    \end{itemize}
  \end{itemize}
\end{itemize}

\subsubsection{Casos de teste derivados}\label{casos-de-teste-derivados}

\begin{longtable}[]{@{}
  >{\raggedright\arraybackslash}p{(\linewidth - 10\tabcolsep) * \real{0.0600}}
  >{\raggedright\arraybackslash}p{(\linewidth - 10\tabcolsep) * \real{0.1100}}
  >{\raggedright\arraybackslash}p{(\linewidth - 10\tabcolsep) * \real{0.0800}}
  >{\raggedright\arraybackslash}p{(\linewidth - 10\tabcolsep) * \real{0.1800}}
  >{\raggedright\arraybackslash}p{(\linewidth - 10\tabcolsep) * \real{0.1600}}
  >{\raggedright\arraybackslash}p{(\linewidth - 10\tabcolsep) * \real{0.4100}}@{}}
\toprule\noalign{}
\begin{minipage}[b]{\linewidth}\raggedright
Caso
\end{minipage} & \begin{minipage}[b]{\linewidth}\raggedright
\texttt{str}
\end{minipage} & \begin{minipage}[b]{\linewidth}\raggedright
\texttt{size}
\end{minipage} & \begin{minipage}[b]{\linewidth}\raggedright
\texttt{stringPreenche}
\end{minipage} & \begin{minipage}[b]{\linewidth}\raggedright
Esperado
\end{minipage} & \begin{minipage}[b]{\linewidth}\raggedright
Justificativa
\end{minipage} \\
\midrule\noalign{}
\endhead
\bottomrule\noalign{}
\endlastfoot
1 & \texttt{null} & 5 & ``-'' & \texttt{null} & \texttt{str} é
\texttt{null} \\
2 & \texttt{""} & 3 & ``.'' & ``\ldots{}'' & \texttt{str} vazia,
preenchida completamente \\
3 & ``a'' & 3 & ``-'' & ``--a'' & preenchimento de 2 com 1 caractere \\
4 & ``abc'' & 3 & ``*'' & ``abc'' & tamanho já suficiente \\
5 & ``abc'' & 4 & ``*'' & ``*abc'' & precisa de 1 caractere \\
6 & ``abc'' & 6 & ``-='' & ``-=-abc'' & padrão de preenchimento longo \\
7 & ``abc'' & 5 & \texttt{""} & '' abc'' & stringPreenche vazia →
espaço \\
8 & ``abc'' & 5 & \texttt{null} & '' abc'' & stringPreenche null →
espaço \\
9 & ``abcdef'' & 4 & ``\#'' & ``abcdef'' & já tem tamanho maior que
size \\
10 & ``abc'' & 10 & ``123'' & ``1231231abc'' & múltiplos preenchimentos
+ truncamento \\
11 & ``abc'' & 5 & '' '' & '' abc'' & espaço explícito como
preenchimento \\
\end{longtable}

\subsection{Teste estrutural (MC/DC)}\label{teste-estrutural-mcdc}

\subsubsection{Passo 1: Definir as condições atômicas do
método}\label{passo-1-definir-as-condiuxe7uxf5es-atuxf4micas-do-muxe9todo}

\begin{itemize}
\tightlist
\item
  \textbf{C1:} \texttt{str\ ==\ null} (true/false)\\
\item
  \textbf{C2:} \texttt{stringPreenche\ ==\ null} (true/false)\\
\item
  \textbf{C3:} \texttt{stringPreenche.isEmpty()} (true/false) --- só
  avaliada se \texttt{C2\ =\ false}, mas para a tabela verdade,
  considerada independente\\
\item
  \textbf{C4:} \texttt{fillLength\ \textless{}=\ 0} (true/false), onde
  \texttt{fillLength\ =\ size\ -\ str.length()}
\end{itemize}

Como temos 4 condições binárias, o total será 2⁴ = 16 linhas na tabela
verdade completa.

\subsubsection{Passo 2: Tabela verdade
completa}\label{passo-2-tabela-verdade-completa}

\begin{longtable}[]{@{}
  >{\raggedright\arraybackslash}p{(\linewidth - 12\tabcolsep) * \real{0.1111}}
  >{\raggedright\arraybackslash}p{(\linewidth - 12\tabcolsep) * \real{0.0741}}
  >{\raggedright\arraybackslash}p{(\linewidth - 12\tabcolsep) * \real{0.0741}}
  >{\raggedright\arraybackslash}p{(\linewidth - 12\tabcolsep) * \real{0.0741}}
  >{\raggedright\arraybackslash}p{(\linewidth - 12\tabcolsep) * \real{0.0741}}
  >{\raggedright\arraybackslash}p{(\linewidth - 12\tabcolsep) * \real{0.3704}}
  >{\raggedright\arraybackslash}p{(\linewidth - 12\tabcolsep) * \real{0.2222}}@{}}
\toprule\noalign{}
\begin{minipage}[b]{\linewidth}\raggedright
Caso
\end{minipage} & \begin{minipage}[b]{\linewidth}\raggedright
C1
\end{minipage} & \begin{minipage}[b]{\linewidth}\raggedright
C2
\end{minipage} & \begin{minipage}[b]{\linewidth}\raggedright
C3
\end{minipage} & \begin{minipage}[b]{\linewidth}\raggedright
C4
\end{minipage} & \begin{minipage}[b]{\linewidth}\raggedright
Resultado esperado
\end{minipage} & \begin{minipage}[b]{\linewidth}\raggedright
Observação
\end{minipage} \\
\midrule\noalign{}
\endhead
\bottomrule\noalign{}
\endlastfoot
T1 & 1 & 0 & 0 & 0 & retorna null & str null \\
T2 & 1 & 0 & 0 & 1 & retorna null & str null \\
T3 & 1 & 0 & 1 & 0 & retorna null & str null \\
T4 & 1 & 0 & 1 & 1 & retorna null & str null \\
T5 & 1 & 1 & 0 & 0 & retorna null & str null \\
T6 & 1 & 1 & 0 & 1 & retorna null & str null \\
T7 & 1 & 1 & 1 & 0 & retorna null & str null \\
T8 & 1 & 1 & 1 & 1 & retorna null & str null \\
T9 & 0 & 0 & 0 & 0 & preenche com stringPreenche & Caso normal
preenche \\
T10 & 0 & 0 & 0 & 1 & retorna str & fillLength \textless= 0 \\
T11 & 0 & 0 & 1 & 0 & preenche com espaço & stringPreenche.isEmpty() =
true \\
T12 & 0 & 0 & 1 & 1 & retorna str & fillLength \textless= 0 \\
T13 & 0 & 1 & 0 & 0 & preenche com espaço & stringPreenche == null \\
T14 & 0 & 1 & 0 & 1 & retorna str & fillLength \textless= 0 \\
T15 & 0 & 1 & 1 & 0 & preenche com espaço & stringPreenche == null e
isEmpty \\
T16 & 0 & 1 & 1 & 1 & retorna str & fillLength \textless= 0 \\
\end{longtable}

\subsubsection{Passo 3: Explicação dos pares de independência para
MC/DC}\label{passo-3-explicauxe7uxe3o-dos-pares-de-independuxeancia-para-mcdc}

O critério MC/DC exige que cada condição atômica:

\begin{itemize}
\tightlist
\item
  Afete a decisão final
\item
  Seja demonstrada por pares de casos de teste onde apenas aquela
  condição muda e as outras permanecem iguais
\end{itemize}

\subsubsection{Passo 4: Seleção de pares independentes para cada
condição}\label{passo-4-seleuxe7uxe3o-de-pares-independentes-para-cada-condiuxe7uxe3o}

\begin{itemize}
\tightlist
\item
  \textbf{C1:} Pares (T8, T16)
\item
  \textbf{C2:} Pares (T9, T13)
\item
  \textbf{C3:} Pares (T9, T11)
\item
  \textbf{C4:} Pares (T9, T10)
\end{itemize}

\subsubsection{Passo 5: Tabela final reduzida para
MC/DC}\label{passo-5-tabela-final-reduzida-para-mcdc}

\begin{longtable}[]{@{}llllll@{}}
\toprule\noalign{}
Caso & C1 & C2 & C3 & C4 & Resultado esperado \\
\midrule\noalign{}
\endhead
\bottomrule\noalign{}
\endlastfoot
T8 & 1 & 1 & 1 & 1 & retorna null \\
T16 & 0 & 1 & 1 & 1 & retorna str \\
T9 & 0 & 0 & 0 & 0 & preenche com stringPreenche \\
T13 & 0 & 1 & 0 & 0 & preenche com espaço \\
T11 & 0 & 0 & 1 & 0 & preenche com espaço \\
T10 & 0 & 0 & 0 & 1 & retorna str \\
\end{longtable}

\subsubsection{Passo 6: Testes JUnit para esses
casos}\label{passo-6-testes-junit-para-esses-casos}

\begin{Shaded}
\begin{Highlighting}[]
\AttributeTok{@Test}
\DataTypeTok{void} \FunctionTok{testStrNull}\OperatorTok{()} \OperatorTok{\{}
    \FunctionTok{assertNull}\OperatorTok{(}\FunctionTok{preenche}\OperatorTok{(}\KeywordTok{null}\OperatorTok{,} \DecValTok{5}\OperatorTok{,} \StringTok{"{-}"}\OperatorTok{));}
\OperatorTok{\}}

\AttributeTok{@Test}
\DataTypeTok{void} \FunctionTok{testPreencheComStringPreenche}\OperatorTok{()} \OperatorTok{\{}
    \FunctionTok{assertEquals}\OperatorTok{(}\StringTok{"***abc"}\OperatorTok{,} \FunctionTok{preenche}\OperatorTok{(}\StringTok{"abc"}\OperatorTok{,} \DecValTok{6}\OperatorTok{,} \StringTok{"*"}\OperatorTok{));}
\OperatorTok{\}}

\AttributeTok{@Test}
\DataTypeTok{void} \FunctionTok{testPreencheComEspacoStringPreencheNull}\OperatorTok{()} \OperatorTok{\{}
    \FunctionTok{assertEquals}\OperatorTok{(}\StringTok{"  abc"}\OperatorTok{,} \FunctionTok{preenche}\OperatorTok{(}\StringTok{"abc"}\OperatorTok{,} \DecValTok{5}\OperatorTok{,} \KeywordTok{null}\OperatorTok{));}
\OperatorTok{\}}

\AttributeTok{@Test}
\DataTypeTok{void} \FunctionTok{testPreencheComEspacoStringPreencheEmpty}\OperatorTok{()} \OperatorTok{\{}
    \FunctionTok{assertEquals}\OperatorTok{(}\StringTok{"  abc"}\OperatorTok{,} \FunctionTok{preenche}\OperatorTok{(}\StringTok{"abc"}\OperatorTok{,} \DecValTok{5}\OperatorTok{,} \StringTok{""}\OperatorTok{));}
\OperatorTok{\}}

\AttributeTok{@Test}
\DataTypeTok{void} \FunctionTok{testRetornaStrFillLengthZero}\OperatorTok{()} \OperatorTok{\{}
    \FunctionTok{assertEquals}\OperatorTok{(}\StringTok{"abcdef"}\OperatorTok{,} \FunctionTok{preenche}\OperatorTok{(}\StringTok{"abcdef"}\OperatorTok{,} \DecValTok{4}\OperatorTok{,} \StringTok{"\#"}\OperatorTok{));}
\OperatorTok{\}}
\end{Highlighting}
\end{Shaded}

\subsection{Análise de cobertura com
Jacoco}\label{anuxe1lise-de-cobertura-com-jacoco}

A cobertura de código foi medida utilizando a ferramenta Jacoco em dois
momentos:

\subsubsection{Sem os testes MC/DC}\label{sem-os-testes-mcdc}

\begin{itemize}
\tightlist
\item
  \textbf{Instruções cobertas:} 93\% (40 de 43 instruções)
\item
  \textbf{Branches cobertos:} 100\% (10 de 10 branches)
\item
  \textbf{Linhas cobertas:} 12 de 13 linhas
\item
  \textbf{Métodos cobertos:} 2 de 2 métodos
\item
  \textbf{Classes cobertas:} 1 de 1 classe
\end{itemize}

\subsubsection{Com os testes MC/DC}\label{com-os-testes-mcdc}

\begin{itemize}
\tightlist
\item
  \textbf{Instruções cobertas:} 93\% (40 de 43 instruções)
\item
  \textbf{Branches cobertos:} 100\% (10 de 10 branches)
\item
  \textbf{Linhas cobertas:} 12 de 13 linhas
\item
  \textbf{Métodos cobertos:} 2 de 2 métodos
\item
  \textbf{Classes cobertas:} 1 de 1 classe
\end{itemize}

\subsubsection{Conclusão}\label{conclusuxe3o}

O critério MC/DC foi aplicado com sucesso, identificando as condições
atômicas relevantes e garantindo que cada uma seja testada de forma
independente. Os casos selecionados cobrem todas as combinações
necessárias para garantir o correto funcionamento da função
\texttt{preenche}. A ferramenta \textbf{Jacoco} pode ser usada para
validar a cobertura dos testes.




\end{document}
